\documentclass[11pt]{memoir}
\usepackage{longtable}
\usepackage{color}
\usepackage{tabu}
\usepackage{setspace}
\usepackage{pdflscape}
\usepackage{graphicx}
%\usepackage{subfigure}
\usepackage{caption}
\usepackage{subcaption}
\usepackage{natbib}
\usepackage{fullpage}
%\bibliographystyle{plain}
\bibliographystyle{cbe}
\usepackage{algorithmic}
\usepackage[vlined,ruled]{algorithm2e}
\usepackage{amsmath}
\usepackage{amsfonts}
\usepackage{amssymb}
\usepackage{url}

\let \oriAlgorithm=\algorithm%
\renewcommand{\algorithm}[2]{%
	\oriAlgorithm%
	\SetAlgoLined%
	\DontPrintSemicolon%
	\SetKwComment{Comment}{$\triangleright\ $}{}%
	\TitleOfAlgo{ #1 }%
	\caption{ #2 }%
}%
\newcommand*{\Set}[2]{ #1 $\gets$ #2 \;}
\newcommand*{\SetC}[3]{ #1 $\gets$ #2 \Comment*[r]{#3}}
\newcommand*{\ForC}[3]{\For(\Comment*[f]{#2}){#1}{#3}}

\begin{document}
	%\firstpage{1}
	
	\title{Documentation for mergeDistMats Program}
	\author{Ward C. Wheeler\\
		Division of Invertebrate Zoology,\\ American Museum of Natural History,\\ 200 Central Park West, New York, NY, 10024, USA;\\wheeler@amnh.org}
	
	
	\maketitle
	\begin{center}
		Running Title: mergeDistMats
	\end{center}
	\newpage
	
	
	\section{Introduction}
	This is the first version of documentation for the program mergeDistMats.  This program is designed to 
	merge phylogenetic distance matrices for subsequent input to phylogenetic distance analysis programs sch as 
	Wag2020 (\url{https://githib.com/wardwheeler/wag2020}).\\ \\
	All source code, precompiled binaries, test data, and documentation are available from \url{https://githib.com/wardwheeler/mergeDistMats}.\\
	
	\noindent This first version is brief.
	
	\section{Input Distance Matrix Format}
	Input distance matrices should be in CVS format (can be exported from spreadsheet programs),  with first line of
	taxon (terminal) names and square matrix following.  First column (from second row) contains distances, not repeat of names.\\
	
	\noindent Matrices must be symmetrical and non-negative, but may have different (if overlapping) taxon (terminal) sets.\\
	
	\noindent Beware of the MS-Excel
	non-conforming CVS output.  A final line feed needs to be added to Excel created CVS files.
	
	
	\section{Output Formats}
	The program outputs (to stdout) a unified matrix in CVS format.
	
	\section{Command options}
	There is a single command line option to be specified.  Either ``normalize'' or `weight''.  This parameter specifies
	whether the input matrices (and output matrix) are scaled on [0,1] (``normalize'') to force an equal weighting on each input matrix.  
	The alternative, ``weight'' leaves the input values as they are and simply adds them together. \\

	

	\noindent The program requires  a single argument and at least one input matrix file.

	
	\section{Program Use}
	The program is invoked from the command-line as in:\\
	\\
	mergeDistMats normalize testData.csv testData2.csv $>$ ouputMatrix.csv\\ \\
	mergeDistMats weight testData.csv testData2.csv $>$ ouputMatrix.csv\\ \\
	
	\subsection{Execution in Parallel}
	By default the program will execute using a single process core.  By specifying the options `+RTS -NX -RTS' where `X' is the number of processors offered to the program. These are specified after the program as in (for 4 parallel threads):\\
	\\
	mergeDistMats +RTS -N4 -RTS other options...  \\
	
	\section*{Acknowledgments}
	The author would like to thank NSF REU DBI-1358465, DARPA SIMPLEX N66001-15-C-4039, and Robert J. Kleberg Jr. and Helen C. Kleberg foundation grant ``Mechanistic Analyses of Pancreatic Cancer Evolution'' for financial support.  
	
	%\newpage
	%\bibliography{big-refs-3.bib}
	%\bibliography{/users/ward/Dropbox/Work_stuff/manus/big-refs-3.bib}
	%\bibliography{/home/ward/Dropbox/Work_stuff/manus/big-refs-3.bib}
\end{document}